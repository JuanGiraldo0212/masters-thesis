% Enable referencing paragraphs
% From https://tex.stackexchange.com/a/7628/46854
\setcounter{secnumdepth}{6}

% Define globally used colors
\definecolor{linkcolor}{rgb}{0.0,0.0,0.55}
\definecolor{refcolor}{rgb}{0.0,0.42,0.24}

% Use footnotes inside table and tabular environments
% From: https://tex.stackexchange.com/a/109471/46854
\makesavenoteenv{tabular}
\makesavenoteenv{table}

% Color links
\hypersetup{
	% These options produce an error:
	% backref     = true,
	% pagebackref = true,
	%
	colorlinks  = true,
	linkcolor   = linkcolor,
	anchorcolor = linkcolor,
	citecolor   = refcolor,
	filecolor   = linkcolor,
	urlcolor    = linkcolor
}

% Change the page title for the ToC
% From: https://tex.stackexchange.com/a/539977/46854
\renewcaptionname{english}{\contentsname}{Table of Contents}

% Header and footer setup
\pagestyle{headings}

% Define util commands to reduce boilerplate
\newcommand\startchapter[1]{\stepcounter{mtc}\chapter{#1}\thispagestyle{empty}}
\newcommand\startappendix[1]{\chapter{#1}\thispagestyle{empty}}
\newcommand\startfirstchapter[1]{\stepcounter{mtc}\chapter{#1}}
\newcommand\myphantomsection[2]{
	\thispagestyle{plain}
	\phantomsection
	\addcontentsline{toc}{chapter}{#1}
	#2
	\clearpage}

% Format the figure number (Figure 1.|: the caption)
\renewcommand{\figureformat}{\figurename~\thefigure\autodot|}

% Remove period after section number in the Table of Contents
% From: https://tex.stackexchange.com/a/223855/46854
\BeforeStartingTOC{\def\autodot{}}

% Remove period after part number (e.g., Part I.)
% From: https://tex.stackexchange.com/a/102313
\renewcommand*{\partformat}{\partname~\thepart}

% Enlarge chapter titles but keep the chapter label and number of the same size
% Also, remove period after chapter number (e.g., Chapter 1.)
\definecolor{chlabelcolor}{rgb}{0.25,0.25,0.25}
\renewcommand*{\chapterformat}{%
	\normalfont\fontsize{18pt}{18pt}\selectfont\color{chlabelcolor}%
	\chapapp~\thechapter\enskip}
\addtokomafont{chapter}{\normalfont\fontsize{26pt}{26pt}\selectfont}

% Prefix parts
% From: https://tex.stackexchange.com/a/358371/46854
\xpatchcmd\addparttocentry
{\addtocentrydefault{part}{#1}{#2}}
{\ifstr{#1}{}
	{\addtocentrydefault{part}{#1}{#2}}%
	{\addtocentrydefault{part}{}{\partname\nobreakspace\texorpdfstring{#1\autodot\enskip}{}#2}}%
}
{}{}
\RedeclareSectionCommand[tocnumwidth=0pt]{part}

% Prefix chapters
\xpatchcmd\addchaptertocentry
{\addtocentrydefault{chapter}{#1}{#2}}
{\ifstr{#1}{}
	{\addtocentrydefault{chapter}{#1}{#2}}%
	{\addtocentrydefault{chapter}{}{\hspace*{1.5em}\chapapp\nobreakspace\texorpdfstring{#1\autodot\enskip}{}#2}}%
}
{}{}
\RedeclareSectionCommand[tocnumwidth=0pt]{chapter}

% Indent sections, subsections, and subsubsections
\setlength\cftsecindent{3.0em}
\setlength\cftsubsecindent{4.4em}
\setlength\cftsubsubsecindent{5.8em}

% Add vertical space after the table of contents inside chapters
\let\oldminitoc\minitoc
\renewcommand{\minitoc}[0]{\oldminitoc\vspace{1em}}

% Move the table of contents inside chapters to the left (reset indentation)
% From: https://tex.stackexchange.com/a/268206/46854
\mtcsetoffset{minitoc}{-1.5em}
\setlength{\mtcindent}{0em}

% Increase the depth of the table of contents inside chapters to display subsubsections
% From: https://tex.stackexchange.com/a/348281/46854
\setcounter{minitocdepth}{3}

% Float Customization
\renewcommand{\floatpagefraction}{0.01}

% Customization of Tables of Contents and List of Figures/Tables (package tocloft)
\renewcommand\cfttabpresnum{Table\ }
\renewcommand\cfttabnumwidth{0.75in}
\renewcommand\cftfigpresnum{Figure\ }
\renewcommand\cftfignumwidth{0.80in}
\newcommand{\HRule}{\rule{\linewidth}{0.5mm}}

% Setup old style numerals macro
\newcommand\osn[1]{\oldstylenums{#1}\xspace}

% Increase table cell padding
\renewcommand*\arraystretch{1.1}

% Squeeze a little some spaces
% More options: https://cs.dartmouth.edu/~dfk/advice/latex-squeeze.html
\setstretch{1.1}
\setlength{\parindent}{1.5em}

% Make the glossary
\makenoidxglossaries
% Enable links to the acronyms page
\glsenablehyper

% Cleveref section format (e.g., Sections 1.3-5)
% From: https://tex.stackexchange.com/a/155951
\crefrangelabelformat{section}{#3#1#4--#5\crefstripprefix{#1}{#2}#6}

% Correct bug when using tocloft and minitoc
% Section titles should be bold in minitoc's table of contents
% From: https://tex.stackexchange.com/a/300547/46854
\renewcommand{\cftsecfont}{\small\bfseries}

% Style description environments
\setlist[description]{font=\hspace{1.5em}\normalfont}

% Define a fake citation command to have arbitrary text as citations
\newcommand{\fakecite}[3]{%
\hypersetup{linkcolor=refcolor}\nocite{#1}%
[\hyperlink{#2}{#3}]\hypersetup{linkcolor=linkcolor}}

% Create invisible hyperlinks over figures (used on CH1 - Fig 1.1.)
% From https://tex.stackexchange.com/a/111451/46854
\newcommand\invlink[4]{%
	\node at (a.center)
	[anchor=center,xshift=#1,yshift=#2]
	% Comment out/uncomment these lines to uncover/cover the link on the figure
	{\hyperref[#3]{\phantom{#4}}};
%	{\hyperref[#3]{#4}};
}

% Circle around letter
% Another alternative from: https://tex.stackexchange.com/a/123926/46854
\newcommand\encircle[1]{\textcircled{\footnotesize #1}}

% Colored circle around letter
%
\definecolor{myyellow}{HTML}{ffc200}
\definecolor{myblack}{HTML}{262626}
\definecolor{myblue}{HTML}{007DFF}
\definecolor{mypurple}{HTML}{7030a0}
% Draw a circled char
\NewDocumentCommand{\encirclecolor}{ O{myblack} O{white} m }{
	\tikz[baseline=(char.base)]{
		\node[shape=circle,inner sep=0.9pt,fill=#1,text=#2] (char) {\ttfamily\bfseries\footnotesize#3};
	}
}
\newcommand\encircleblack[1]{\encirclecolor{#1}}
\newcommand\encircleblue[1]{\encirclecolor[myblue][white]{#1}}
\newcommand\encirclepurple[1]{\encirclecolor[mypurple][white]{#1}}
\newcommand\encircleyellow[1]{\encirclecolor[myyellow][black]{#1}}
\newcommand\encirclered[1]{\encirclecolor[red][white]{#1}}
\NewDocumentCommand{\encirclecolord}{ O{myblack} O{white} O{myyellow} m }{
	\tikz[baseline=(char.base)]{
		\node[shape=circle,inner sep=0.9pt,fill=#3,text=#3] at (0.15, 0) (char) {\ttfamily\bfseries\footnotesize#4};
		\node[shape=circle,inner sep=0.9pt,fill=#1,text=#2] at (0, 0) (char) {\ttfamily\bfseries\footnotesize#4};
	}
}

% Create a paragraph followed by a new line
\newcommand{\startparagraph}[1]{\paragraph{#1}\mbox{}\\}

% Simulate a new line break
\newcommand\mylinebreak{\\[10pt]}

% Setup algorithm environments
% comments
\newcommand\commentsfont[1]{\footnotesize\ttfamily\textcolor{refcolor}{#1}}
\SetCommentSty{commentsfont}
% line numbers
\newcommand\numbersfont[1]{\normalsize\ttfamily\textcolor{gray}{#1}}
\SetNlSty{numbersfont}{}{}


% Short right arrow
\makeatletter
\newcommand\shortto[0]{%
	\mathrel{\mathpalette\short@to\relax}%
}
\newcommand{\short@to}[2]{%
	\mkern2mu
	\clipbox{{.4\width} 0 0 0}{$\m@th#1\vphantom{+}{\shortrightarrow}$}%
}
\makeatother
\newcommand\shortra[0]{\rightarrow}

% Double-sided arrow with text at the top and at the bottom
% Adapted from: https://tex.stackexchange.com/a/51176/46854
% Super tiny font from: https://tex.stackexchange.com/a/57678/46854
\makeatletter
\newcommand\xleftrightarrow[2][]{\ext@arrow 0099{\longleftrightarrowfill@}{\scalebox{.5}{#1}}{\scalebox{.5}{#2}}}
\def\longleftrightarrowfill@{\arrowfill@\leftarrow\relbar\rightarrow}
\makeatother

% Setup function plots (evaluation chapter)
\pgfplotsset{compat=newest, every tick label/.append style={font=\tiny}}

% Small bullet (evaluation chapter)
\newcommand{\sbt}{\,\begin{picture}(-1,1)(-1,-3)\circle*{2}\end{picture}\ ~}

% Listings
\definecolor{keywords}{RGB}{127,0,85} % 145,13,93
\definecolor{code}{RGB}{90,90,90}
\definecolor{extensions}{RGB}{185,44,39}
\definecolor{comments}{RGB}{63,127,95}
\definecolor{strings}{RGB}{74,0,255}
\definecolor{frame}{RGB}{230,230,230}
\definecolor{numbers}{RGB}{160,160,160}
\definecolor{linebg}{RGB}{240,240,240}
\definecolor{consolebg}{RGB}{21,45,53}
\definecolor{consoletext}{RGB}{200,200,200}
\definecolor{consolekeyword}{RGB}{186,50,189}
\definecolor{consolegreen}{RGB}{88,230,55}

\lstdefinestyle{base}{
	tabsize=2,
	captionpos=b,
	showspaces=false,
	showtabs=false,
	breaklines=true,
	showstringspaces=false,
	breakatwhitespace=true,
	escapeinside={(*@}{@*)},
	commentstyle=\color{comments},
	keywordstyle=\bfseries\color{keywords},
	keywordstyle=[2]\color{extensions},
	stringstyle=\color{strings},
	basicstyle=\footnotesize\ttfamily,
	frame=lines,
	rulecolor=\color{frame},
	xleftmargin=1.2em,
	framexleftmargin=1.2em,
	numbers=left,
	numbersep=4pt,
	numberstyle=\footnotesize\ttfamily\color{numbers},
	aboveskip=1em,
}

\lstdefinestyle{json}{
	style=base,
	language=java,
	literate=
	*{0}{{{\color{numbers}0}}}{1}
	{1}{{{\color{numbers}1}}}{1}
	{2}{{{\color{numbers}2}}}{1}
	{3}{{{\color{numbers}3}}}{1}
	{4}{{{\color{numbers}4}}}{1}
	{5}{{{\color{numbers}5}}}{1}
	{6}{{{\color{numbers}6}}}{1}
	{7}{{{\color{numbers}7}}}{1}
	{8}{{{\color{numbers}8}}}{1}
	{9}{{{\color{numbers}9}}}{1}
	{:}{{{\color{keywords}{:}}}}{1}
	{,}{{{\color{keywords}{,}}}}{1}
	{\{}{{{\color{code}{\{}}}}{1}
	{\}}{{{\color{code}{\}}}}}{1}
	{[}{{{\color{code}{[}}}}{1}
	{]}{{{\color{code}{]}}}}{1},
}

\lstdefinestyle{console}{
	style=base,
	language=bash,
	basicstyle=\color{consoletext}\footnotesize\ttfamily,
	keywordstyle=\bfseries\color{consolekeyword},
	frame=none,
	frameround=tttt,
	framesep=5em,
	morekeywords={
		ForkAndCollectAlgorithm,
		RuntimeAgent,
		Server,
		TerraformRepository,
		CamClient
	},
}
% Rounded corners - console
% From: https://tex.stackexchange.com/a/27676/46854
% Documentation: https://tools.ietf.org/doc/texlive-doc/latex/mdframed/mdframed-example-default.pdf
\global\mdfdefinestyle{consolestyle}{%
	leftmargin=0pt,
	rightmargin=0pt,
	middlelinewidth=0pt,
	backgroundcolor=consolebg,
	fontcolor=consoletext,
	roundcorner=4
}

\lstdefinestyle{xml}{
	style=base,
	language=xml,
	morekeywords={
		augmentation,
		input,
		output,
		mapping,
		mappings,
		monitors,
		monitor,
		transformation
	},
}

\lstdefinestyle{iac}{
	style=base,
	morekeywords={
		variable,
		output,
		provider,
		data,
		resource,
		clone,
		customize,
		linux\_options,
		network\_interface,
		disk,
		var
	},
	literate=
	*{\$}{{{\color{extensions}\$}}}{1}
	{.}{{{\color{extensions}{.}}}}{1}
	{:}{{{\color{extensions}{:}}}}{1}
	{,}{{{\color{extensions}{,}}}}{1}
	{\{}{{{\color{extensions}{\{}}}}{1}
	{\}}{{{\color{extensions}{\}}}}}{1}
	{[}{{{\color{extensions}{[}}}}{1}
	{]}{{{\color{extensions}{]}}}}{1}
}

% mdframed environment for the "Challenges" box in CH 2
\newenvironment{InfoBox}[1][]{%
	\ifstrempty{#1}%
	{\mdfsetup{%
			frametitle={%
				\tikz[baseline=(current bounding box.east),outer sep=0pt]
				\node[line width=0.3pt,anchor=east,rectangle,draw=black!100,fill=black!3]
				{\strut Information};}}
	}%
	{\mdfsetup{%
			frametitle={%
				\tikz[baseline=(current bounding box.east),outer sep=0pt]
				\node[line width=0.3pt,anchor=east,rectangle,draw=black!100,fill=black!3]
				{\strut #1};}}%
	}%
	\mdfsetup{frametitleaboveskip=\dimexpr-\ht\strutbox\relax,linewidth=0.3pt}
	\begin{mdframed}[]\relax%
	}{\end{mdframed}}

% i.e., e.g., cf., et al.
\newcommand\ie[1]{\textit{i.e.}, #1}
\newcommand\eg[1]{\textit{e.g.}, #1}
\newcommand\cf[1]{\textit{cf.} #1}
\newcommand\etal[0]{\textit{et al.}}

% Correct bad hyphenation
\hyphenation{Cyber-Physical}
\sloppy

% Title page stuff
%
% lək̓ʷəŋən (Lekwungen) but as a figure to prevent any encoding issues
%  Latex is incapable of processing these characters and I don't want to
%  use XeLatex or something similar
\newcommand{\lekwungen}{%
	\begingroup\normalfont%
	\begin{minipage}{1.18cm}%
		\includegraphics[height=1.359\fontcharht\font`\B]{fig/frontmatter/lekwungen.pdf}%
	\end{minipage}%
	\endgroup%
}
% Draw a horizontal bar below a character, similar to W̱SÁNEĆ (Saanich)
\newcommand\barbelow[1]{\stackunder[1.2pt]{#1}{\rule{.8ex}{.1ex}}}
