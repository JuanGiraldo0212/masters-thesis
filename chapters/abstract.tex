\myphantomsection{Abstract}{

% \chapter* does not set a head mark!
\addchap*{Abstract}

Increasingly complex dynamics in the software operations pose formidable software evolution challenges to the software industry. Examples of these dynamics include the globalization of software markets, the massive increase of interconnected devices worldwide with the internet of things, and the digital transformation to large-scale cyber-physical systems. To tackle these challenges, researchers and practitioners have developed impressive bodies of knowledge, including adaptive and autonomic systems, run-time models, continuous software engineering, and the practice of combining software development and operations (\ie{DevOps}). Despite the tremendous strides the software engineering community has made toward managing highly dynamic systems, software-intensive industries face major challenges to match the ever-increasing pace. To cope with this rapid rate at which operational contexts for software systems change, organizations are required to automate and expedite software evolution on both the development and operations sides.

The aim of our research is to develop continuous and autonomic methods, infrastructures, and tools to realize software evolution holistically. In this dissertation, we shift the prevalent autonomic computing paradigm and provide new perspectives and foci on integrating autonomic computing techniques into continuous software engineering practices, such as DevOps. Our methods and approaches are based on online experimentation and evolutionary optimization. Experimentation allows autonomic managers to make informed data-driven and explainable decisions and present evidence to stakeholders. As a result, autonomic managers contribute to the continuous and holistic evolution of design, configuration and deployment artifacts, providing guarantees on the validity, quality and effectiveness of enacted changes. Ultimately, our approach turns autonomic managers into online stakeholders whose contributions are subject to quality control.

Our contributions are threefold. We focus on effecting long-lasting software changes through self-management, self-improvement, and self-regulation. First, we propose a framework for continuous software evolution pipelines for bridging offline and online evolution processes. Our framework's infrastructure captures run-time changes and turns them into configuration and deployment code updates. Our functional validation on cloud infrastructure management demonstrates its feasibility and soundness. It effectively contributes to eliminate technical debt from the \glsentryfull{iac} life cycle, allowing development teams to embrace the benefits of \acrshort[hyper=false]{iac} without sacrificing existing automation. Second, we provide a comprehensive implementation for the continuous \acrshort[hyper=false]{iac} evolution pipeline. Third, we design a feedback loop to conduct experimentation-driven continuous exploration of design, configuration and deployment alternatives. Our experimental validation demonstrates its capacity to enrich the software architecture with additional components, and to optimize the computing cluster’s configuration, both aiming to reduce service latency. Our feedback loop frees DevOps engineers from incremental improvements, and allows them to focus on long-term mission-critical software evolution changes. Fourth, we define a reference architecture to support short-lived and long-lasting evolution actions at run-time. Our architecture incorporates short-term and long-term evolution as alternating autonomic operational modes. This approach keeps internal models relevant over prolonged system operation, thus reducing the need for additional maintenance. We demonstrate the usefulness of our research in case studies that guide the designs of cloud management systems and a Colombian city transportation system with historical data.

In summary, this dissertation presents a new approach on how to manage software continuity and continuous software improvement effectively. Our methods, infrastructures, and tools constitute a new platform for short-term and long-term continuous integration and software evolution strategies and processes for large-scale intelligent cyber-physical systems. This research is a significant contribution to the long-standing challenges of easing continuous integration and evolution tasks across the development-time and run-time boundary. Thus, we expand the vision of autonomic computing to support software engineering processes from development to production and back. This dissertation constitutes a new holistic approach to the challenges of continuous integration and evolution that strengthens the causalities in current processes and practices, especially from execution back to planning, design, and development.

}% phantom section
